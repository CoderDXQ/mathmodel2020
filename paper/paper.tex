% !Mode:: "TeX:UTF-8"
%!TEX program  = xelatex

\documentclass[bwprint]{gmcmthesis}
%\documentclass{cumcmthesis}
%\documentclass[withoutpreface,bwprint]{cumcmthesis} %去掉封面与编号页,电子版提交的时候使用。

\usepackage[framemethod=TikZ]{mdframed}
\title{全降低汽油精制过程中的辛烷值损失模型}


%参赛信息
\baominghao{20100130029} %参赛队号
\schoolname{北京邮电大学}%学校名称
\membera{唐麒淳} %队员AB
\memberb{段祥卿} %队员B
\memberc{戴维} %队员C


% \usepackage{enumitem}
% \setlist[enumerate]{listparindent=\parindent}
\begin{document}


 %生成标题
 \maketitle


 %填写摘要
\begin{abstract}
摘要文字,请删除我巴拉巴拉


问题一:唐麒淳

问题二:唐麒淳

问题三:唐麒淳

问题四:唐麒淳

问题五:唐麒淳

%填写关键字
\keywords{分布转换\quad 基于模型的特征筛选\quad 贝叶斯优化\quad 5折交叉验证\quad TPE算法}
\end{abstract}

%视情况决定是否需要建立新页面
%\newpage


%设置页码???
\pagestyle{plain}

%目录 不推荐加
%\tableofcontents

%生成目录
\tableofcontents
\newpage


%第一章节
\section{问题重述}
\subsection{问题背景}
巴拉巴拉一堆话
\subsection{问题提出}

\textbf{问题1:巴拉巴拉几个字}

巴拉巴拉一堆话巴拉巴拉一堆话巴拉巴拉一堆话巴拉巴拉一堆话巴拉巴拉一堆话巴拉巴拉一堆话巴拉巴拉一堆话巴拉巴拉一堆话巴拉巴拉一堆话巴拉巴拉一堆话巴拉巴拉一堆话巴拉巴拉一堆话巴拉巴拉一堆话巴拉巴拉一堆话巴拉巴拉一堆话巴拉巴拉一堆话巴拉巴拉一堆话巴拉巴拉一堆话巴拉巴拉一堆话


\textbf{问题2:}

\textbf{问题3:}

\textbf{问题4:}

\textbf{问题5:}


%第二章节
\section{模型假设}

%有序列表 这里写论文的模型假设
%左侧缩进可能还需要调整
\begin{enumerate}[itemindent=20pt]
	\item 假设一
	\item 假设二
\end{enumerate}


\section{符号说明}

\begin{table}[htbp]
    \centering
    \caption{论文中用到的符号定义}
    \begin{tabular}{ccccc}
        \toprule
        \multicolumn{5}{c}{符号} \\
        \midrule
        amsbsy & amsfonts & {amsgen} & {amsmath} & {amsopn} \\
        amssymb & amstext & {appendix} & {array} & {atbegshi} \\

        \bottomrule
    \end{tabular}%
    \label{tab:addlabel}%
\end{table}%



\section{问题分析与求解}
\subsection{引言}

创意平板折叠桌注重于表达木制品的优雅和设计师所想要强调的自动化与功能性。为了增大有效使用面积。设计师以长方形木板的宽为直径截取了一个圆形作为桌面,又将木板剩余的面积切割成了若干个长短不一的木条,每根木条的长度为平板宽到圆上一点的距离,分别用两根钢筋贯穿两侧的木条,使用者只需提起木板的两侧,便可以在重力的作用下达到自动升起的效果,相互对称的木条宛如下垂的桌布,精密的制作工艺配以质朴的木材,让这件工艺品看起来就像是工业革命时期的机器。

\subsection{问题的提出}

\subsubsection{问题的提出内容一}

围绕创意平板折叠桌的动态变化过程、设计加工参数,本文依次提出如下问题:

(1)给定长方形平板尺寸 ($120 cm \times 50 cm \times 3 cm$),每根木条宽度(2.5 cm),连接桌腿木条的钢筋的位置,折叠后桌子的高度(53 cm)。要求建立模型描述此折叠桌的动态变化过程,并在此基础上给出此折叠桌的设计加工参数和桌脚边缘线的数学描述。



(2)折叠桌的设计应做到产品稳固性好、加工方便、用材最少。对于任意给定的折叠桌高度和圆形桌面直径的设计要求,讨论长方形平板材料和折叠桌的最优设计加工参数,例如,平板尺寸、钢筋位置、开槽长度等。对于桌高70 cm,桌面直径80 cm的情形,确定最优设计加工参数。


(3)给出软件设计的数学模型,可以根据客户任意设定的折叠桌高度、桌面边缘线的形状大小和桌脚边缘线的大致形状,给出所需平板材料的形状尺寸和切实可行的最优设计加工参数,使得生产的折叠桌尽可能接近客户所期望的形状,并根据所建立的模型给出几个设计的创意平板折叠桌。要求给出相应的设计加工参数,画出至少8张动态变化过程的示意图。

\section{模型的假设}

\begin{itemize}
\item 忽略实际加工误差对设计的影响;
\item 木条与圆桌面之间的交接处缝隙较小,可忽略;
\item 钢筋强度足够大,不弯曲;
\item 假设地面平整。
\end{itemize}

\section{符号说明}

\begin{tabular}{cc}
 \hline
 \makebox[0.4\textwidth][c]{符号}	&  \makebox[0.5\textwidth][c]{意义} \\ \hline
 D	    & 木条宽度(cm) \\ \hline
 L	    & 木板长度(cm)  \\ \hline
 W	    & 木板宽度(cm)  \\ \hline
 N	    & 第n根木条  \\ \hline
 T	    & 木条根数  \\ \hline
 H	    & 桌子高度(cm)  \\ \hline
 R	    & 桌子半径(cm)  \\ \hline
 R	    & 桌子直径(cm)  \\ \hline
\end{tabular}

\section{问题分析}

\subsection{问题一分析}
题目要求建立模型描述折叠桌的动态变化图,由于在折叠时用力大小的不同,我们不能描述在某一时刻折叠桌的具体形态,但我们可以用每根木条的角度变化来描述折叠桌的动态变化。首先,我们知道折叠桌前后左右对称,我们可以运用几何知识求出四分之一木条的角度变化。最后,根据初始时刻和最终形态两种状态求出桌腿木条开槽的长度。



\subsection{问题二分析}
题目要求从折叠桌的稳固性好、加工方便、用材最少三个角度,确定设计加工参数。我们可以从应力、支撑面积考虑稳固性,从开槽长度考虑加工方便,从木板长度考虑用材最少。而它们之间又是相互制约,我们需要确定最优设计加工参数,可以建立非线性规划模型,用lingo软件来求解最优设计加工参数(平板尺寸、钢筋位置、开槽长度等),这里以合力的方向(斜向上)与最长木条(桌腿)的夹角方向最小为目标函数,以木条所承受应力小于木条的许用应力、支撑面积大于桌面面积、木条的开槽长度小于木条本身长为约束条件。
\begin{figure}[!h]
\centering
\includegraphics[width=.7\textwidth]{1.png}
\caption{问题三流程图}
\end{figure}
\subsection{问题三分析}
题目要求制作软件的意思就是客户给定折叠桌高度、桌面边缘线的形状大小和桌脚边缘线的大致形状,将这些信息输入程序就得到客户想要的桌子。我们在求解最优设计加工参数时,自行给定桌面边缘线形状(椭圆、相交圆等),桌脚边缘线形状,折叠桌高度,应用第二问的非线性规划模型,用MATLAB软件绘制折叠桌截面图,得到自己设计的创意平板折叠桌。



% 参考文献   手工录入
% \begin{thebibliography}{9}%宽度9
% \bibitem{bib:one} ....
% \bibitem{bib:two} ....
% \end{thebibliography}


\begin{thebibliography}{99}  
\bibitem{ref1}Zheng L, Wang S, Tian L, et al., Query-adaptive late fusion for image search and person re-identification, Proceedings of the IEEE Conference on Computer Vision and Pattern Recognition, 2015: 1741-1750.  
\bibitem{ref2}Arandjelović R, Zisserman A, Three things everyone should know to improve object retrieval, Computer Vision and Pattern Recognition (CVPR), 2012 IEEE Conference on, IEEE, 2012: 2911-2918.  
\bibitem{ref3}Lowe D G. Distinctive image features from scale-invariant keypoints, International journal of computer vision, 2004, 60(2): 91-110.  
\bibitem{ref4}Philbin J, Chum O, Isard M, et al. Lost in quantization: Improving particular object retrieval in large scale image databases, Computer Vision and Pattern Recognition, 2008. CVPR 2008, IEEE Conference on, IEEE, 2008: 1-8.  
\end{thebibliography}


% %采用bibtex方案
% \cite{mittelbach_latex_2004,wright_latex3_2009,beeton_unicode_2008,vieth_experiences_2009}

% \bibliographystyle{gmcm}
% \bibliography{example}


%参考文献 2019年大学生数学建模比赛的模板中提取的
\begin{thebibliography}{9}%宽度9
    \bibitem{1}{liuhaiyang2013latex}
    刘海洋.
    \newblock \LaTeX {}入门\allowbreak[J].
    \newblock 电子工业出版社, 北京, 2013.
    \bibitem{2}{mathematical-modeling}
    全国大学生数学建模竞赛论文格式规范 (2020 年 8 月 25 日修改).
    \bibitem{3} \url{https://www.latexstudio.net}
\end{thebibliography}


\newpage
%附录
\appendix
%\setcounter{page}{1} %如果需要可以自行重置页码。
\section{我的 MATLAB 源程序}
\begin{lstlisting}[language=Matlab]%设置不同语言即可。
kk=2;[mdd,ndd]=size(dd);
while ~isempty(V)
[tmpd,j]=min(W(i,V));tmpj=V(j);
for k=2:ndd
[tmp1,jj]=min(dd(1,k)+W(dd(2,k),V));
tmp2=V(jj);tt(k-1,:)=[tmp1,tmp2,jj];
end
tmp=[tmpd,tmpj,j;tt];[tmp3,tmp4]=min(tmp(:,1));
if tmp3==tmpd, ss(1:2,kk)=[i;tmp(tmp4,2)];
else,tmp5=find(ss(:,tmp4)~=0);tmp6=length(tmp5);
if dd(2,tmp4)==ss(tmp6,tmp4)
ss(1:tmp6+1,kk)=[ss(tmp5,tmp4);tmp(tmp4,2)];
else, ss(1:3,kk)=[i;dd(2,tmp4);tmp(tmp4,2)];
end;end
dd=[dd,[tmp3;tmp(tmp4,2)]];V(tmp(tmp4,3))=[];
[mdd,ndd]=size(dd);kk=kk+1;
end; S=ss; D=dd(1,:);


 \end{lstlisting}


\end{document} 