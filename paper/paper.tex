% !Mode:: "TeX:UTF-8"
%!TEX program  = xelatex

\documentclass[bwprint]{gmcmthesis}
%\documentclass{cumcmthesis}
%\documentclass[withoutpreface,bwprint]{cumcmthesis} %去掉封面与编号页,电子版提交的时候使用。

\usepackage[framemethod=TikZ]{mdframed}
%\usepackage{hyperref}
%\usepackage{tocloft}
\title{全降低汽油精制过程中的辛烷值损失模型}


%参赛信息
\baominghao{20100130029} %参赛队号
\schoolname{北京邮电大学}%学校名称
\membera{唐麒淳} %队员AB
\memberb{段祥卿} %队员B
\memberc{戴维} %队员C


% \usepackage{enumitem}
% \setlist[enumerate]{listparindent=\parindent}
\begin{document}


 %生成标题
 \maketitle


 %填写摘要
\begin{abstract}
摘要文字,请删除我巴拉巴拉


问题一:唐麒淳

问题二:唐麒淳

问题三:唐麒淳

问题四:唐麒淳

问题五:戴维

%填写关键字
\keywords{分布转换\quad 基于模型的特征筛选\quad 贝叶斯优化\quad 5折交叉验证\quad TPE算法}
\end{abstract}

%视情况决定是否需要建立新页面
%\newpage


%设置页码???
\pagestyle{plain}

%目录 不推荐加
%\tableofcontents

%生成目录
 \tableofcontents

\newpage


%第一章节
\section{问题重述}
\subsection{问题背景}
巴拉巴拉一堆话
\subsection{问题提出}

\textbf{问题1:巴拉巴拉几个字}

巴拉巴拉一堆话巴拉巴拉一堆话巴拉巴拉一堆话巴拉巴拉一堆话巴拉巴拉一堆话巴拉巴拉一堆话巴拉巴拉一堆话巴拉巴拉一堆话巴拉巴拉一堆话巴拉巴拉一堆话巴拉巴拉一堆话巴拉巴拉一堆话巴拉巴拉一堆话巴拉巴拉一堆话巴拉巴拉一堆话巴拉巴拉一堆话巴拉巴拉一堆话巴拉巴拉一堆话巴拉巴拉一堆话


\textbf{问题2:}

\textbf{问题3:}

\textbf{问题4:}

\textbf{问题5:}


%第二章节
\section{模型假设}

%有序列表 这里写论文的模型假设
%左侧缩进可能还需要调整
\begin{enumerate}[itemindent=20pt]
	\item 假设一
	\item 假设二
\end{enumerate}


%第三章节
\section{符号说明}

\begin{table}[htb]
    \centering
    \caption{论文中用到的符号定义}
	\begin{tabular}{cc}
	\toprule
	 \makebox[0.4\textwidth][c]{符号}	&  \makebox[0.5\textwidth][c]{意义} \\ 
	 \midrule
	 D	    & 木条宽度(cm)  \\
	 N	    & 第n根木条  \\ 
	   \midrule
	 T	    & 木条根数  \\
	 H	    & 桌子高度(cm)  \\
	 \bottomrule
	\end{tabular}
    \label{tab:addlabel}%
\end{table}%


% \begin{tabular}{cc}
% \toprule
%  \makebox[0.4\textwidth][c]{符号}	&  \makebox[0.5\textwidth][c]{意义} \\ 
%  D	    & 木条宽度(cm)  \\
%  L	    & 木板长度(cm)  \\
%  W	    & 木板宽度(cm)  \\
%  N	    & 第n根木条  \\ 
%    \midrule
%  T	    & 木条根数  \\
%  H	    & 桌子高度(cm)  \\
%  R	    & 桌子半径(cm)  \\
%  R	    & 桌子直径(cm)  \\

%  \bottomrule
% \end{tabular}



%第四章节
\section{问题分析与求解}
\subsection{问题一:数据处理}
\subsubsection{问题分析}

由于(套话)原因,采集到的原始数据存在一些异常,包括(套话)


问题一的目标是参考近4年的工业数据(325个数据样本数据.xlsx),依“样本确定方法”对285号和313号数据样本进行预处理,并加入到原工业数据中。

\subsubsection{对含空值变量的分析}

处理1:对时序上基本平稳的含空值变量做均值填充

我们在313样本的数据中,发现了2个时序上基本平稳但含有空值的变量

TODO: 两图并列


\begin{figure}[ht]
	\centering
	\includegraphics[width=.7\textwidth]{313-fill-1}
	\caption{S-ZORB.FT\_1204.PV}
\end{figure}


\begin{figure}[ht]
	\centering
	\includegraphics[width=.7\textwidth]{313-fill-2}
	\caption{D-121含硫污水排量}
\end{figure}

对于这两个变量的空值用313样本数据的其余采样相应变量的均值来填充

处理2:对时序上杂乱或呈周期趋势的含空值变量做删除操作

TODO: 三图并列


\begin{figure}[]
	\centering
	\includegraphics[width=.7\textwidth]{313-del-1}
	\caption{D-106热氮气流量}
\end{figure}


\begin{figure}[ht]
	\centering
	\includegraphics[width=.7\textwidth]{313-del-2}
	\caption{S-ZORB.FT\_2431.DACA}
\end{figure}


\begin{figure}[ht]
	\centering
	\includegraphics[width=.7\textwidth]{313-del-3}
	\caption{3.0步骤FIC2432.SP}
\end{figure}



\subsubsection{剔除不在操作范围内的样本}

考虑到“325个样本数据.xlsx”中的很多样本已经超出了“354个操作变量信息.xlsx”所规定的范围,我们进行了简单的处理,用“325个样本数据”中每个操作变量的最大与最小值来扩充“354个操作变量信息”中操作变量的范围,并用扩充后的操作变量范围代替原操作变量范围。

经过计算,285样本数据中所有的采样数据都在操作变量范围内,而313样本数据仅有1个采样数据的所有变量在操作变量范围内。经过综合考虑,我们决定用以下公式来计算一个采样样本操出操作变量范围的程度:

\begin{equation}
	InvalidDegree=\sum^M_j{\frac{Exceed_j}{Upper_j - Lower_j}}
	\label{eq:range-invalid-degree}
\end{equation}

其中,$j$表示某采样样本的第$j$个操作变量,$M$表示共有$M$个操作变量, $Upper_j$表示操作变量$j$的上界,$Lower_j$表示操作变量$j$的下界,$Exceed_j$表示操作变量$j$超出范围的大小。

通过上述公式,我们对313样本的采样数据进行了计算:

\begin{table}[ht]
	\caption{313样本超出范围程度}\label{tab:001} \centering
	\begin{tabular}{ccc}
	\toprule[1.5pt]
	time &  invalid-degree &  rank \\
	\midrule[1pt]
	2017-05-15 06:57:00 &        1.986283 &     0 \\
	2017-05-15 07:24:00 &        1.868216 &     1 \\
	2017-05-15 07:18:00 &        1.533825 &     2 \\
	2017-05-15 06:33:00 &        1.514671 &     3 \\
	2017-05-15 06:54:00 &        1.298304 &     4 \\
	2017-05-15 06:51:00 &        1.057390 &     5 \\
	2017-05-15 07:21:00 &        0.876193 &     6 \\
	2017-05-15 07:51:00 &        0.866200 &     7 \\
	2017-05-15 07:48:00 &        0.770276 &     8 \\
	2017-05-15 07:54:00 &        0.658065 &     9 \\
	\bottomrule[1.5pt]
\end{tabular}
\end{table}


我们决定以$InvalidDegree >= 1$为阈值,删除满足其条件的所有采样样本




\subsubsection{用3$\sigma$准则删除异常样本}

3$\sigma$准则:设对被测量变量进行等精度测量,得到$x_1, x_2, \ldots, x_n$,算出其算术平均值$x$及剩余误差$v_i=x_i-x (i=1, 2, \ldots , n)$,并按贝塞尔公式算出标准误差$\sigma$,若某个测量值$x_b$的剩余误差$v_b ( 1 <= b <= n )$,满足 $|v_b| = | x_b - x | > 3\sigma$,则认为$x_b$是含有粗大误差值的坏值,应予剔除。贝塞尔公式如下:

\begin{equation}\label{eq:3sigma}
	\sigma = [\frac{1}{n-1}\sum_{i=1}^{n}v_i^2]^{\frac{1}{2}} = \{\frac{[\sum^n_{i=1}x_i^2 - (\sum^n_{i=1}x_i)^2/n]}{n-1}\}^{\frac{1}{2}}
\end{equation}

根据上述公式,285样本的所有采样都满足3$\sigma$准则,而313样本有27个采样不满足其条件,故删除这些不满足条件的变量。



\subsubsection{对工业数据中的空值变量进行处理}

通过上述操作,将313样本和285样本经过处理后加入到325个样本的工业数据中,并将4.1.2处理2(引用)中删除的变量设置为 $NaN$ ,待进一步的处理。

考虑到工业设备巴拉巴拉(套话),我们认为0代表了工业数据中的空值。对于不同类型的情况,我们制定了3种处理空值的策略:

\begin{equation}\label{eq:empty-strategies}
\left\{
\begin{aligned}
delete \ column, \ & if \ len(EmptyElements)>50 \ and \ mean(vector)>5 \\  
delete \ element, \ & if \ len(EmptyElements)<50 \ and \ mean(vector)>5\\  
do \ not \ process, \ & if \ mean(vector)<=5
\end{aligned}
\right.
\end{equation}

根据公式(引用)中,如果某列空值元素的长度 $len(EmptyElements)$ 大于50,并且这列元素的均值 $ mean(vector)$ 大于5,说明空值较多,做空值填充的意义不大,应该将此列删除。如果某列空值元素的长度小于50,并且这列元素的均值大于5,说明空值相对较少,可以通过做空值填充保留下来。如果这列元素的均值小于等于5,说明这列元素基本为0,0可能不是这列元素的空值,所以不做处理。

\begin{figure}[ht]
	\centering
	\includegraphics[width=.7\textwidth]{strategy-1}
	\caption{策略1}
\end{figure}


\begin{figure}[ht]
	\centering
	\includegraphics[width=.7\textwidth]{strategy-2}
	\caption{策略2}
\end{figure}


\begin{figure}[ht]
	\centering
	\includegraphics[width=.7\textwidth]{strategy-3}
	\caption{策略3}
\end{figure}

\subsection{问题二:寻找建模主要变量}
\subsubsection{问题分析}

由于催化裂化汽油精制过程是连续的,虽然操作变量每3 分钟就采样一次,但辛烷值(因变量)的测量比较麻烦,一周仅2次无法对应。但根据实际情况可以认为辛烷值的测量值是测量时刻前两小时内操作变量的综合效果,因此预处理中取操作变量两小时内的平均值与辛烷值的测量值对应。这样产生了325个样本(见附件一)。

建立降低辛烷值损失模型涉及包括7个原料性质、2个待生吸附剂性质、2个再生吸附剂性质、2个产品性质等变量以及另外354个操作变量(共计367个变量),工程技术应用中经常使用先降维后建模的方法,这有利于忽略次要因素,发现并分析影响模型的主要变量与因素。因此,请你们根据提供的325个样本数据(见附件一),通过降维的方法从367个操作变量中筛选出建模主要变量,使之尽可能具有代表性、独立性(为了工程应用方便,建议降维后的主要变量在30个以下),并请详细说明建模主要变量的筛选过程及其合理性。(提示:请考虑将原料的辛烷值作为建模变量之一)。

\subsubsection{通过特征筛选获取主要变量}

\begin{figure}[htbp]
	\centering
	\includegraphics[width=.7\textwidth]{feature-selection-perfs}
	\caption{特征筛选n次后剩余特征数}
\end{figure}

\begin{figure}[htbp]
	\centering
	\includegraphics[width=.7\textwidth]{feature-selection-feats}
	\caption{特征筛选次数与$r^2$评价指标的关系}
\end{figure}

\begin{table}[!htbp]
	\caption{特征筛选n次后的各指标}\label{tab:001} \centering
	\begin{tabular}{ccc}
		\toprule[1.5pt]
	feature selection times &  $r^2$ metrics &  n\_features left \\
		\midrule[1pt]
           0 &       0.101955 &              347 \\
			1 &       0.140157 &               96 \\
			2 &       0.184434 &               28 \\
			3 &       0.217998 &               12 \\
			4 &       0.095470 &                6 \\
			5 &       0.053913 &                3 \\
		\bottomrule[1.5pt]
	\end{tabular}
\end{table}

\subsubsection{主要变量数据分析}

\begin{table}[!htbp]
	\caption{主要变量的特征重要度}\label{tab:001} \centering
	\begin{tabular}{ccccc}
		\toprule[1.5pt]
	 rank & feature importances & percentage (\%) &                  name &        CN name \\
		\midrule[1pt]
    0 &              0.0785 &          12.83 &  S-ZORB.PDT\_1003.DACA &  P-101B入口过滤器差压 \\
	1 &              0.0774 &          12.66 &    S-ZORB.PC\_1001A.PV &    D101原料缓冲罐压力 \\
	2 &              0.0602 &           9.84 &     S-ZORB.TC\_5005.PV &        稳定塔下部温度 \\
	3 &              0.0596 &           9.74 &   S-ZORB.LI\_9102.DACA &        D-204液位 \\
	4 &              0.0515 &           8.42 &   S-ZORB.TE\_1107.DACA &  E-101D壳程出口管温度 \\
	5 &              0.0474 &           7.75 &    S-ZORB.SIS\_TE\_2802 &        D-102温度 \\
	6 &              0.0443 &           7.25 &     S-ZORB.TE\_5202.PV &      精制汽油出装置温度 \\
	7 &              0.0408 &           6.67 &   S-ZORB.TE\_1605.DACA &  F-101出口支管\#4温度 \\
	8 &              0.0390 &           6.37 &   S-ZORB.DT\_2001.DACA &    R-101下部床层压降 \\
	9 &              0.0385 &           6.29 &     S-ZORB.PC\_1603.PV &   加热炉主火嘴瓦斯入口压力 \\
	10 &              0.0379 &           6.20 &     S-ZORB.AT\_5201.PV &     精制汽油出装置硫含量 \\
	11 &              0.0364 &           5.95 &                   RON &            辛烷值 \\
		\bottomrule[1.5pt]
	\end{tabular}
\end{table}

\begin{figure}[htbp]
	\centering
	\includegraphics[width=.7\textwidth]{feat-imp}
	\caption{主要变量的特征重要度}
\end{figure}

\begin{figure}[htbp]
	\centering
	\includegraphics[width=.7\textwidth]{pairplot}
	\caption{相关性矩阵图}
\end{figure}

\begin{figure}[htbp]
	\centering
	\includegraphics[width=.7\textwidth]{heatmap}
	\caption{热力图}
\end{figure}

\subsection{问题三:建立辛烷值(RON)损失预测模型}
\subsubsection{问题分析}

采用上述样本和建模主要变量,通过数据挖掘技术建立辛烷值(RON)损失预测模型,并进行模型验证。 

\subsubsection{删除异常样本}

\begin{figure}[htbp]
	\centering
	\includegraphics[width=.7\textwidth]{RON-loss-abnormal}
	\caption{删除异常样本}
\end{figure}


\subsubsection{模型训练与验证}

\begin{figure}[htbp]
	\centering
	\includegraphics[width=.7\textwidth]{RON-loss-cross-validation}
	\caption{对RON-loss进行交叉验证}
\end{figure}


\begin{figure}[htbp]
	\centering
	\includegraphics[width=.7\textwidth]{S-cross-validation}
	\caption{对S进行交叉验证}
\end{figure}

\subsection{问题四:主要变量操作方案的优化}

\subsubsection{问题分析}

要求在保证产品硫含量不大于5μg/g的前提下,利用你们的模型获得325个数据样本(见附件四“325个数据样本数据.xlsx”)中,辛烷值(RON)损失降幅大于30%的样本对应的主要变量优化后的操作条件(优化过程中原料、待生吸附剂、再生吸附剂的性质保持不变,以它们在样本中的数据为准)。


\subsection{问题五:模型的可视化展示}
\subsubsection{问题分析}

工业装置为了平稳生产,优化后的主要操作变量(即:问题2中的主要变量)往往只能逐步调整到位,请你们对133号样本(原料性质、待生吸附剂和再生吸附剂的性质数据保持不变,以样本中的数据为准),以图形展示其主要操作变量优化调整过程中对应的汽油辛烷值和硫含量的变化轨迹。(各主要操作变量每次允许调整幅度值Δ见附件四“354个操作变量信息.xlsx”)。

%第五章节
%下面这块看看怎么改
\section{模型评价}


\subsection{模型的优点}
巴拉巴拉一堆话巴拉巴拉一堆话巴拉巴拉一堆话巴拉巴拉一堆话巴拉巴拉一堆话巴拉巴拉一堆话巴拉巴拉一堆话巴拉巴拉一堆话巴拉巴拉一堆话巴拉巴拉一堆话巴拉巴拉一堆话巴拉巴拉一堆话巴拉巴拉一堆话巴拉巴拉一堆话巴拉巴拉一堆话巴拉巴拉一堆话巴拉巴拉一堆话巴拉巴拉一堆话巴拉巴拉一堆话



\subsection{模型的缺点}
巴拉巴拉一堆话巴拉巴拉一堆话巴拉巴拉一堆话巴拉巴拉一堆话巴拉巴拉一堆话巴拉巴拉一堆话巴拉巴拉一堆话巴拉巴拉一堆话巴拉巴拉一堆话巴拉巴拉一堆话巴拉巴拉一堆话巴拉巴拉一堆话巴拉巴拉一堆话巴拉巴拉一堆话巴拉巴拉一堆话巴拉巴拉一堆话巴拉巴拉一堆话巴拉巴拉一堆话巴拉巴拉一堆话






\section{加图的方法,记得删除}
\begin{figure}[!h]
\centering
\includegraphics[width=.7\textwidth]{f1.png}
\caption{问题三流程图}
\end{figure}



%下面两种参考文献的格式二选一

%参考文献重新分页
\newpage

% 参考文献   手工录入
% \begin{thebibliography}{9}%宽度9
% \bibitem{bib:one} ....
% \bibitem{bib:two} ....
% \end{thebibliography}


\begin{thebibliography}{99}  
\bibitem{ref1}Zheng L, Wang S, Tian L, et al., Query-adaptive late fusion for image search and person re-identification, Proceedings of the IEEE Conference on Computer Vision and Pattern Recognition, 2015: 1741-1750.  
\bibitem{ref2}Arandjelović R, Zisserman A, Three things everyone should know to improve object retrieval, Computer Vision and Pattern Recognition (CVPR), 2012 IEEE Conference on, IEEE, 2012: 2911-2918.  
\bibitem{ref3}Lowe D G. Distinctive image features from scale-invariant keypoints, International journal of computer vision, 2004, 60(2): 91-110.  
\bibitem{ref4}Philbin J, Chum O, Isard M, et al. Lost in quantization: Improving particular object retrieval in large scale image databases, Computer Vision and Pattern Recognition, 2008. CVPR 2008, IEEE Conference on, IEEE, 2008: 1-8.  
\end{thebibliography}


% %采用bibtex方案
% \cite{mittelbach_latex_2004,wright_latex3_2009,beeton_unicode_2008,vieth_experiences_2009}

% \bibliographystyle{gmcm}
% \bibliography{example}


%参考文献 2019年大学生数学建模比赛的模板中提取的
\begin{thebibliography}{9}%宽度9
    \bibitem{1}{liuhaiyang2013latex}
    刘海洋.
    \newblock \LaTeX {}入门\allowbreak[J].
    \newblock 电子工业出版社, 北京, 2013.
    \bibitem{2}{mathematical-modeling}
    全国大学生数学建模竞赛论文格式规范 (2020 年 8 月 25 日修改).
    \bibitem{3} \url{https://www.latexstudio.net}
\end{thebibliography}






\newpage
%附录
\appendix
%\setcounter{page}{1} %如果需要可以自行重置页码。
\section{我的 Python 源程序}
\begin{lstlisting}[language=Python]%设置不同语言即可。
kk=2;[mdd,ndd]=size(dd);
while ~isempty(V)
[tmpd,j]=min(W(i,V));tmpj=V(j);
for k=2:ndd
[tmp1,jj]=min(dd(1,k)+W(dd(2,k),V));
tmp2=V(jj);tt(k-1,:)=[tmp1,tmp2,jj];
end
tmp=[tmpd,tmpj,j;tt];[tmp3,tmp4]=min(tmp(:,1));
if tmp3==tmpd, ss(1:2,kk)=[i;tmp(tmp4,2)];
else,tmp5=find(ss(:,tmp4)~=0);tmp6=length(tmp5);
if dd(2,tmp4)==ss(tmp6,tmp4)
ss(1:tmp6+1,kk)=[ss(tmp5,tmp4);tmp(tmp4,2)];
else, ss(1:3,kk)=[i;dd(2,tmp4);tmp(tmp4,2)];
end;end
dd=[dd,[tmp3;tmp(tmp4,2)]];V(tmp(tmp4,3))=[];
[mdd,ndd]=size(dd);kk=kk+1;
end; S=ss; D=dd(1,:);


 \end{lstlisting}


\end{document} 